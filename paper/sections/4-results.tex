\section{Results}

\subsection{Metrics}

Image segmentation models are mostly evaluated on the basis of how accurately they can predict each pixel. The prediction of the pixels can be true positive (TP), true negative (TN) or false positive (FP) and false negative (FN) respectively. The model’s performance is determined by the nature of its prediction and the aforementioned category it belongs to. We plan to use the pixel accuracy to decide the specificity, sensitivity or accuracy of our prediction.

\par
\emph{Sensitivity} measures the proportion of the correctly identified positives and is calculated as

\begin{equation}
  Sensitivity = \frac{TP}{TP + FN}
\end{equation}

\emph{Specificity} tells us the proportion of the correctly identified negatives and can be calculated as

\begin{equation}
  Specificity = \frac{TN}{TN + FP}
\end{equation}

For a given prediction, \emph{accuracy} is defined as

\begin{equation}
  Accuracy = \frac{TP + TN}{TP + FP + TN + FN}
\end{equation}

Sensitivity and specificity helps us in calculation of the \emph{F1-score} which is very similar to the Dice coefficient. Using the calculation above, the F1-score is given as

\begin{equation}
  F_1 = \frac{TP}{2 * TP + FN + FP}
\end{equation}

where as the \emph{Dice coefficient} is calculated as

\begin{equation}
  Dice = \frac{2 * TP}{2 * TP + FN + FP}
\end{equation}

We have used the popular \emph{Intersection over Union (IoU)} metric, also referred to as the \emph{Jaccard index}, as a method to quantify the percentage of overlap between the target mask and our prediction output.

\begin{equation}
  J(A, B) = \frac{|A \cup B|}{|A \cap B|}
\end{equation}

This metric is closely related to the Dice coefficient which is often used as a loss function during training. Quite simply, the IoU metric measures the number of pixels common between the target and prediction masks divided by the total number of pixels present across both masks.

\begin{equation}
  J(A, B) = \frac{TP}{TP + FN + FP}
\end{equation}


The 2018 lesion segmentation challenge defines the \emph{threshold Jaccard Index}, averaged over all images in the dataset, as the primary evaluation criteria where the threshold is set to 0.65 \citep{challenge-2018-codella}. The threshold Jaccard Index can be calculated as

\begin{equation}
  TJ(A, B) = \begin{cases}
      J(A, B), & \text{if}\ J(A, B) \geq{0.65} \\
      0, & \text{otherwise}
    \end{cases}
\end{equation}

We use the same specific evaluation metrics for our work.


\subsection{Metrics}

\subsubsection{Training Environment Setup}

All experiments were conducted using the same VM running Linux 18.04.1-Ubuntu (x86\_64). The underlying hardware contained 54 GiB physical memory, 64-bits Intel(R) Xeon(R) CPU E5-2690 v3 (2.60GHz) and NVIDIA Tesla K80 (8GK210GL, rev a1) GPU with 12 GiB physical memory.

% \begin{itemize}
%   \item Describe all Machine Learning Theory techniques used to evaluate the success of the model.
%   \item Describe how results compare to previous research.
% \end{itemize}
